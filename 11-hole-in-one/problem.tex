\documentclass[12pt]{report}
\usepackage[a4paper, total={7.3in, 9.7in}]{geometry}
\usepackage{amsmath}
\usepackage{upquote}
\usepackage{listings}
\usepackage{xcolor}
\usepackage{titlesec}
\usepackage{amssymb}

\definecolor{backgroundcolor}{rgb}{1, 1, 1}
\definecolor{commentstyle}{rgb}{0.365, 0.422, 0.475}
\definecolor{keywordstyle}{rgb}{0.6, 0.14, 0.576}
\definecolor{numberstyle}{rgb}{0.5, 0.5, 0.5}
\definecolor{stringstyle}{rgb}{0.77, 0.1, 0.08}

\lstdefinestyle{xcodecolor}{
    backgroundcolor=\color{backgroundcolor},   
    commentstyle=\color{commentstyle},
    keywordstyle=\color{keywordstyle},
    numberstyle=\scriptsize\color{numberstyle},
    stringstyle=\color{stringstyle},
    basicstyle=\ttfamily\footnotesize,
    breakatwhitespace=false,         
    breaklines=true,                 
    captionpos=b,                    
    keepspaces=true,                   
    numbersep=5pt,                  
    showspaces=false,                
    showstringspaces=false,
    showtabs=false,                  
    tabsize=2
}

\lstset{style=xcodecolor}

\usepackage[T1]{fontenc}
\usepackage{cascadia-code}
\usepackage{hyperref}

% Raised Rule Command:
%  Arg 1 (Optional) - How high to raise the rule
%  Arg 2            - Thickness of the rule
\newcommand{\raisedrule}[2][0em]{\leaders\hbox{\rule[#1]{1pt}{#2}}\hfill}

\setlength{\parindent}{0pt}
\titleformat{\section}
{\normalfont\Large\bfseries}{\thesection}{1em}{}[{\titlerule[0.8pt]}]

\begin{document}

	{\Large
	\textbf{Hole In One}}
	
	\vspace{0.4cm}
	DiPS CodeJam 24\raisedrule[0.25em]{1pt}
	\\
	% document

	\section*{Prompt}
	In the game of Golf, you are given a 2D grid representing the golf course. Each cell of the grid can have one of the following values:

	0: An empty cell.\\
	1: A cell with a golf hole.\\
	2: A cell with an obstacle.\\
	
	You start at a given cell on the grid and need to determine if it's possible to reach any golf hole with a single swing of the golf club. A swing is defined as moving from the starting cell in a straight line (either horizontally, vertically, or diagonally) until you either hit an obstacle, the edge of the grid, or reach a golf hole.

	Given an $m$ by $n$ grid and a starting position, can you see if a hole-in-one is possible?

	\subsection*{Input Format}
	\begin{itemize}
		\item The first line of the input contains 4 space separated integers $m$ $n$ $x$ $y$, denoting an $m$ by $n$ grid and a starting point of $(x,y)$ such that \texttt{grid[x][y]} is possible.
		\item The next $m$ lines contain $n$ space separated integers denoting one row of the grid.
	\end{itemize}
	\subsection*{Output Format}
	The first and only line of your output must contain a single integer $h$, 1 if hole-in-one is possible and 0 if not.
	\subsection*{Constraints}
	\begin{itemize}
		\item $ 10 \le m,n \le 100 $
	\end{itemize}

	\subsection*{Sample Input/Output}
	\begin{tabular}{ |l|l| } 
		\hline
		\textbf{Input} & \textbf{Output} \\
		% use {\lstinputlisting{./testCases/input/input00.txt}} & {\lstinputlisting{./testCases/output/output00.txt}} \\
		\hline
	\end{tabular}

\section*{Sample Program}
	\lstinputlisting[language=Python]{sampleSolution.py}
	

\end{document}